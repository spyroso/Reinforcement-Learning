\documentclass[a4paper,12pt]{article}

% Packages
\usepackage{float}
\usepackage[utf8]{inputenc}
\usepackage{fancyhdr}
\usepackage[left = 0.6 in, right = 0.6 in, top = 1 in, bottom = 1 in, headsep = 0.5 in]{geometry}
\usepackage[normalem]{ulem}
\usepackage{enumerate}
\usepackage{pgfplots}
\usepackage{titling}
\pgfplotsset{width=8cm,compat=1.9}
\usepackage{stackengine}
%\pagenumbering{gobble}
\usepackage[fleqn]{amsmath}
\usepackage{amssymb}
\usepackage[T1]{fontenc}
\usepackage{fancybox}

\hfuzz = 100pt

%%% Horizontal Line Command
\makeatletter
  \newcommand*\variableheghtrulefill[1][.4\p@]{%
    \leavevmode
    \leaders \hrule \@height #1\relax \hfill
    \null
  }
\makeatother
%%%

%%% Header & Footer
\fancyhf{}
\renewcommand{\footrulewidth}{0.1mm}
\fancyhead[L]{Bakr Abbas, Matteo Esposito, Spyros Orfanos, Frederic Siino}
%\fancyhead[C]{Assignment 1}
\fancyhead[R]{STAT 497-H}
%\fancyfoot[L]{ID: 40024121, 40025959, 400XXXXX}
%\fancyfoot[C]{Concordia University
\fancyfoot[R]{\thepage}

\pagestyle{fancy}
%%%

% Code formatting
\def\code#1{\texttt{#1}}

\renewcommand\maketitlehooka{\null\mbox{}\vfill}
\renewcommand\maketitlehookd{\vfill\null}

% Custom settings
\setlength{\parskip}{0.75em}  % Paragraph spacing
\setcounter{section}{-1} % Page numbers to start on page 1
\setlength\parindent{0pt} % Remove indenting from entire file
\def\layersep{2.5cm}

\title{\textbf{Assignment 1}}
\author{STAT 497-H | Reinforcement Learning}
\date{Bakr Abbas (40000333), Matteo Esposito (40024121), Spyros Orfanos (40032280), Frederic Siino (40028348)}

%%%%%%%%%%%%%%%%%%%%%%%%%%%%%%%%%%%%%%%%%%%%%%%%%%%%%%%%%%%%%%%%%%%%%%%%%%%%%%%%

\begin{document}
\begin{titlingpage}
  \maketitle
  \centering
  \vfill
  {\large{Concordia University}}\par
  {\large{February 1st, 2019}}
\end{titlingpage}

\newpage

\subsection*{Question 1}

\textbf{a)} See \code{main.R} code.

\textbf{b)} See \code{main.R} code.

\textbf{c)}

\begin{figure}[H]
  \centering
  \includegraphics[width=8cm]{figures/q1_avgReward.png}
  \qquad
  \includegraphics[width=8cm]{figures/q1_percOptimal.png}
\end{figure}

From the average reward plot we can observe that initially the greedy action
($\epsilon = 0$) is the best course of action, however this quickly changes
and overtime the less greedy options are clearly better as they promote some 
level of exploration.
\par
From the \% optimal action graph we observe a gradual trend towards a value of 
$1 - \epsilon$ for each curve excluding the $\epsilon = 0$ curve as this is a 
property of the $\epsilon$-greedy approach.

\subsection*{Question 2}

\textbf{a)} See \code{main.R} code.

\textbf{b)} See \code{main.R} code.

\begin{figure}[H]
  \centering
  \includegraphics[width=8cm]{figures/q2_arm1.png}
  \qquad
  \includegraphics[width=8cm]{figures/q2_arm2.png}
\end{figure}

\textbf{c)}
% Which method performs better predictions for the action value? Why?
By observing the graphs of the action value for the two arms, it can be concluded that the 
exponentially weighted method yields more accurate predictions for the action value. Since 
the rewards from both arms follow an autoregressive process, the current reward depends on 
the previous reward. Thus, by using an exponentially weighted step size, we are attributing a 
progressively decreasing weight to older action value estimates and giving more weight to more 
recent rewards, which is more suitable for autoregressive models. Since the equally weighted 
method assigns equal weight to all rewards, the more recent rewards have less of an impact 
than they should, while the older, less relevant rewards, have the same impact as the more
 recent rewards on the estimation.

\subsection*{Question 3}

\textbf{a)} See \code{main.R} code.

\textbf{b)} We see that the probability of choosing Arm 3 approaches 1, which is expected 
since rewards from Arm 3 have the highest mean. Further, the action preference if Arm 2 
is higher than that of Arm 1, which is also expected since its rewards have a higher mean
 that those of Arm 2.

\begin{figure}[H]
  \centering
  \includegraphics[width=12cm]{figures/q3.png}
\end{figure}

\end{document}